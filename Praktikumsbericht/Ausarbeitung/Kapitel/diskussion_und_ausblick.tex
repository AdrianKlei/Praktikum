\chapter{Diskussion und Ausblick}
\label{sec.diskussion_und_ausblick}
In dieser Arbeit erfolgt die Darstellung von Personen-Auftrittswahrscheinlichkeiten mittels einer frequenzbasierten Modellierung. Zu diesem Zweck wird eine Server-Client-Struktur mit spezifischer Aufgabenverteilung geschaffen. Den Ausgangspunkt des Modells bilden dabei Personendetektionen innerhalb eines Belegungsgitters, welches in einzelne Zellen unterteilt wird. Die Zellinformationen werden in einer dezentralen Datenbank in einem standardisierten Format gespeichert. Die Datenbankeinträge enthalten neben dem Zeitstempel des Intervalls die korrespondierende Anzahl an Personendetektionen. Es wird eine Verbindung zwischen dem Client und der Datenbank erstellt. Die Daten werden vom Client eingelesen und je nach betrachtetem Modell, welches dem binären oder dem quantitativen Fall entspricht, aufbereitet. Basierend darauf wird das zugrunde liegende Belegungsgitter rekonstruiert. Die Daten werden in einem weiteren Schritt in einen Trainings- und Testdatensatz eingeteilt und an den Server geschickt. Auf diesem erfolgt eine Untersuchung der Frequenzspektren der Daten. Für jede Zelle des Belegungsgitters wird ein separates Modell erstellt. Mittels der FreMEn-Methode wird eine vorher festgelegte Anzahl an prominenten Frequenzen identifiziert und zur Rekonstruktion der Personendetektions-Daten verwendet. Die Evaluation dieses rekonstruierten Modells erfolgt anhand eines Testdatensatzes, welcher nicht zur Erstellung des Modells verwendet wird. Sowohl für den binären wie auch den quantitativen Fall kann durch die Modellierung der Personen-Auftrittswahrscheinlichkeiten durch eine Superposition periodischer Funktionen eine Genauigkeitssteigerung gegenüber statischen Modellen, welche die Personen-Auftrittswahrscheinlichkeiten als zeitlich konstante Funktion approximieren, nachgewiesen werden. Die Höhe der Genauigkeitssteigerung variiert dabei jedoch stark je nach betrachtetem Datensatz. Für binäre Modelle kann eine Genauigkeitssteigerung von bis zu 9.02 Prozentpunkten erreicht werden (vgl. Tabelle \ref{tab.Prädiktionsfehler aruba_binary}). Es werden jedoch auch Limitationen des Modells sichtbar. Für Zellen, welche eine geringe Anzahl an Personendetektionen vorweisen, können mit einer maximalen Anzahl periodischer Prozesse von $l_\ind{max} = 20$ diese hochfrequenten Ereignisse nicht modelliert werden. Eine weitere Erhöhung der Anzahl periodischer Prozesse resultiert in einem Verlust der Generalisierungsfähigkeit des Modells für Daten, welche nicht zur Modellerstellung verwendet werden. Zellen, für welche beispielsweise nur innerhalb von fünf Prozent aller Zeitintervalle Personendetektionen ermittelt werden, werden durch das statische Modell bereits mit einer Genauigkeit von bis zu 95 \% dargestellt. Eine frequenzbasierte Modellierung der Personen-Auftrittswahrscheinlichkeiten ist demnach nur auf Zellen anwendbar, welche eine gewisse Häufigkeit an Personendetektionen aufweisen. \\
Auch für den quantitativen Fall kann durch eine frequenzbasierte Modellierung der Personenraten $\lambda$ eine Steigerung der Prädiktionsgenauigkeit gegenüber einer statischen Modellierung erreicht werden. Erneut ist die Höhe der Genauigkeitssteigerung stark vom jeweils betrachteten Datensatz abhängig. Gegenüber dem statischen Modell kann der Prädiktionsfehler um bis zu 19.94 \% reduziert werden (vgl. Tabelle \ref{tab.Prädiktionsfehler aruba_float}). \\
Im Hinblick auf die frequenzbasierte Modellierung zur Prädiktion von Auftrittswahrscheinlichkeiten von Personen ergeben sich weiterführende Fragestellungen.
So ist eine zeitliche Aktualisierung des Modells nötig. Einen Lösungsansatz stellt hierfür die Neuberechnung des Modells in festgeschriebenen Intervallen dar. So können die Modelle in einem wöchentlichen Takt unter Berücksichtigung von neuen, vom Roboter aufgezeichneten Daten, aktualisiert werden. Als Trainingsdaten könnten hierfür beispielsweise die Detektionen der letzten Wochen $[n, \dots, n -l]$ benutzt werden. Die Evaluierung erfolgt dann anhand der Daten der letzten $l$ Wochen. \\
Eine weitere Fragestellung bildet die Unvollständigkeit der Detektionsdaten. Für die in dieser Arbeit behandelten Datensätze konnte der mobile Roboter die gesamte Umgebung $\mathcal{U}$ überblicken. Die von ihm aufgezeichneten Informationen des zeitlichen Verlaufes der Belegtheit, bzw. Nichtbelegtheit von Zellen, ist also vollständig. In der Realität kann der Roboter die Umgebung jedoch nicht vollständig überblicken, des Weiteren steht eine dauerhafte statische Positionierung zum Aufnehmen von Personendetektionen im Widerspruch der weiteren Aufgaben eines Service-Roboters. Die Aufenthaltsdauer des Roboters auf dem Umgebungsgebiet ist ungleich verteilt. Für Bereiche mit einem längeren Aufenthalt kann eine genauere Schätzung der Personen-Auftrittswahrscheinlichkeiten abgegeben werden als für Bereiche, in denen sich der Roboter nur selten aufhält. \\
Die Unvollständigkeit der Daten ist durch eine geeignete Modellierung abzubilden. Da für unterschiedliche Umgebungsbereiche durch eine ungleich verteilte Menge an aufgenommenen Daten verschiedene Konfidenzbereiche für die Schätzung der tatsächlichen Personendaten existieren, muss eine Entscheidungsregel gefunden werden, anhand derer ein Wert innerhalb des Konfidenzbereiches als Punktschätzung gewählt wird. Die so ermittelten Daten bilden dann erneut die Eingangsgrößen des in dieser Arbeit vorgestellten Modells ab. 
\chapter{Einleitung}
Mobile Roboter finden immer mehr Einzug in die heutige Zeit. Als Beispiele seien mobile Staubsaug-Roboter oder Rasenmäh-Roboter sowie mobile Serviceroboter, welche mit Menschen interagieren können, zu nennen. All diesen verschiedenen Roboterarten liegt die Gemeinsamkeit zugrunde, dass sie für die effektive Bearbeitung ihrer Aufgaben auf eine modellierte Repräsentation ihrer Umgebung angewiesen sind. Für eine solche Umgebungsrepräsentation bestehen verschiedene Ansätze. Einige dieser Ansätze stellen die Umgebung als statisch dar \cite{Lakemeyer.2003}.  Dieser Ansatz kann für Anwendungsfälle wie einen Rasenmäher-Roboter, welcher wiederholt ein statisches Gebiet abfahren muss, genügen. Eine wesentliche Aufgabe mobiler Serviceroboter ist hingegen die Interaktion mit Personen innerhalb der Umgebung. Die Positionen dieser Personen sind dabei im Allgemeinen nicht statisch, sondern variieren mit der Zeit. Um seinen Aufgaben effektiv nachkommen zu können, benötigt der mobile Serviceroboter ein Modell der zeitlich und örtlich variablen Auftrittswahrscheinlichkeiten von Personen innerhalb seiner Arbeitsumgebung. Mithilfe eines solchen Modells ist der Roboter in der Lage, zu bestimmten Zeitpunkten gezielt Orte mit einem hohen Personenaufkommen anzufahren. Strebt der Roboter eine Kontaktaufnahme an, soll so die Zeit bis zum Erreichen dieser, gegenüber dem Abfahren einer zufälligen Route oder einem Verbleib und dem Warten an der aktuellen Position, reduziert werden. \\
In dieser Arbeit wird die Möglichkeit einer frequenzbasierten Modellierung zur Prädiktion von Personen-Auftrittswahrscheinlichkeiten untersucht. Neben einem binären Modell, welches die Auftrittswahrscheinlichkeiten von Personen prognostiziert, wird ein quantitatives Modell untersucht, welches die Anzahl von Personen innerhalb eines Zeitintervalls vorhersagen soll.  \\
In Kapitel 2 (Grundlagen und Stand der Technik) wird  auf die benötigten mathematischen Grundlagen eingegangen. Es werden Wahrscheinlichkeitsverteilungen und Funktionen vorgestellt, mittels derer binäre und quantitative Zustände modelliert werden können. Darauf aufbauend wird der aktuelle Stand der Technik im Bereich der binären und quantitativen Zustandsmodellierung präsentiert. Ein Hauptbestandteil des Kapitels ist die FreMEn (Frequency Map Enhancement) Methode. Die Methode modelliert Umweltzustände, wie das Personenaufkommen in einer Umgebung, durch zeitlich veränderliche Funktionen. Es erfolgt eine Untersuchung des Frequenzspektrums dieser zeitlich veränderlichen Funktionen und eine Approximation durch periodische, harmonische Funktionen \cite{Krajnik.2014}. \\
% Vllt. noch was zu Jovan schreiben?
In Kapitel 3 (Methodik) wird auf den Modellierungsprozess eingegangen. Es erfolgt eine Beschreibung der Messdatenermittlung und Vorverarbeitung durch ein Belegtheitsgitter, jeweils für binäre Zustandsmodelle sowie quantitative Zustandsmodelle. Im weiteren Verlauf des Kapitels wird die Server-Client-Struktur des Modells beschrieben. Der Client speichert die vorverarbeiteten Daten und schickt sie an den Server, welcher eine Frequenzanalyse der Daten durchführt und die zeitlich veränderlichen Funktionen durch eine Superposition periodischer, harmonischer Funktionen approximiert. Die ermittelten Parameter werden an den Client zurückgesendet, welcher diese speichert. \\
% Sollte ich das noch in das Kapitel 'Evaluation' schreiben? Vllt. mit einer nicen Grafik?
Kapitel 4 (Evaluation) evaluiert das binäre und quantitative Modell mittels zweier Datensätze. Es werden einheitliche Fehlermaße zur Bewertung der Modellgüten definiert. Darauf aufbauend erfolgt ein Vergleich der Modellgüten der binäre und quantitativen Modelle mit statischen Modellen, welche Umweltzustände durch zeitlich invariable, konstante Werte approximieren. Neben Möglichkeiten der Modelle wird auch auf limitierende Faktoren eingegangen. \\
In Kapitel 5 (Zusammenfassung und Ausblick) werden die gesammelten Ergebnisse zusammengefasst. Es erfolgt ein Ausblick auf weiterführende Fragestellungen im Rahmen der frequenzbasierten Modellierung von Personen-Auftrittswahrscheinlichkeiten.